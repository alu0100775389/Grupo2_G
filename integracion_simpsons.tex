\documentclass[spanish,a4paper,11pt]{report}

%%%%%%%%%%%%%%%%%%%%%%%%%%%%%%%%%%%%%%%%%%%%%%%%%%%%%%%%%%%%%%%%%%%%%%%%%%%%%%%
\usepackage[dvips]{graphicx}
\usepackage[dvips]{epsfig}
%\usepackage[latin1]{inputenc}
\usepackage[spanish]{babel}
\usepackage[utf8]{inputenc}
%\usepackage{alltt}
%\usepackage{templates/algorithm}
%\usepackage{templates/algorithmic}
%\usepackage{templates/multirow}
\begin{document}

\pagestyle{empty}
\thispagestyle{empty}


\newcommand{\HRule}{\rule{\linewidth}{1mm}}
\setlength{\parindent}{0mm}
\setlength{\parskip}{0mm}
\vspace*{\stretch{1}}

\begin{center}
%\includegraphics[width=0.2\textwidth]{images/logotipo-secundario-ULL.eps}\\[0.25cm]
\end{center}

\HRule
\begin{center}
        {\Huge \bf Integración Simpsons} \\[2.5mm] 
        {\Huge \Large \[f(x) = \frac{x^3}{1+x^\frac{1}{2}}, x \in [1,2]\]} \\[2.5mm]
        {\Large Lorena Morales Pérez\\[2mm]
                Ashneet Kandhpur Singh\\[2mm]
                Cristina González Marrero}\\[5mm]
        {\Large \textit{Grupo 2 }} \\[5mm]


        {\em Técnicas Experimentales. $1^{er}$ curso. $2^{do}$ semestre} \\[5mm]
        Lenguajes y Sistemas Informáticos \\[5mm]
        Facultad de Matemáticas \\[5mm]
        
        Universidad de La Laguna \\
\end{center}
\HRule
\vspace*{\stretch{2}}
\begin{center}
  La Laguna, \today 
\end{center}


\newpage

\begin{abstract}

\parindent=1cm Sabemos que una integral definida se define geométricamente como el área bajo la curva f(x) en el intervalo[a,b].
Desafortunadamente en la mayoría de los casos prácticos es muy difícil hallar una antiderivada de f(x).
En estos casos el valor de la integraldebe de aproximarse. Existen varias maneras para ello, por modelos ó métodosnuméricos,
aplicar la regla Trapezoidal o Rectangular con segmentos cada vez más pequeños,o por otro lado, utilizar las Reglas de SIMPSON aplicando polinomios de orden superior
para conectar los puntos, con la cual se obtiene una estimación más exacta de una integral. Por ejemplo si hay un punto medio extra entre f(a) y f(b),
entonces se puede conectar los tres puntos con una parábola.
Dicho de otra manera, para cada aplicación de la regla de SIMPSON serequieren dos subintervalos, a fin de aplicarla n número de veces,
deberá dividirse el intervalo (a,b) en un número de subintervalos o segmentos. Cada subintervalo sucesivo se aproxima por un polinomio de segundo grado(parábola)
y se integra de tal manera que la suma de las áreas de cada segmentode la parábola sea la aproximación a la integración deseada.

\end{abstract}

\newpage

\tableofcontents

\newpage

MOTIVACIÓN Y OBJETIVOS

GENERAL

Resolver el problema de cálculo del área bajo la curva entre dos límitesconocidos, dividiendo en N sub áreas para calcular su valor, asumiendo cada subárea como un pequeño arco de parábola.
 
Comprender las bases conceptuales de la integración aproximada 
Comprender los rasgos generales de la integración aproximada utilizandoel método de Simpson 
Comprender la aproximación del error por truncamiento de la integraciónaproximada utilizando el método de Simpson, frente al valor exacto 
Resolver problemas de integración numérica y apreciar su aplicación enla solución de problemas de ingeniería, utilizando el método de Simpson.

ESPECÍFICOS

 
Conocer la interpretación geométrica de la integral definida
Reconocer que el método de Simpson representa, geométricamente, el áreabajo una función polinomial de segundo orden (Cuadrática o Parabólica).
Deducir la fórmula de Simpson a partir de la interpretación geométrica de laintegral definida.
Acotar el error cometido en la integración numérica por el método deSimpson.
Explicar la obtención de fórmulas más precisas para calcular,numéricamente, integrales definidas.
Aplicar el método de Simpson, para calcular numéricamente, lasaproximaciones de algunas integrales definidas.








\end{document}
